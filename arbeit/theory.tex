\chapter{Malware Analysis} \label{chap:analysis}

\begin{definition}
''\emph{Malware analysis} is the art of dissecting malware to understand how it
works, how to identify it, and how to defeat or eliminate it.`` \cite[\p{} xxviii]{sikorski12}
\end{definition} 

\section{Malware Analysis Techniques}

\subsection*{Static Analysis}

\begin{definition}
\emph{Static analysis} is the examination of a program without running it. \cite[\p{} 2]{sikorski12}
\end{definition} 

Static analysis includes \eg{} viewing the file format information, finding strings or patterns of byte sequences, disassembling the program and subsequent examination of the intructions.

\subsection*{Dynamic Analysis}

\begin{definition}
\emph{Dynamic analysis} is the examination of a program while running it. \cite[\p{} 2]{sikorski12}
\end{definition}

Dynamic analysis includes \eg{} observing the program's behaviour in a \VM{} or a dedicated testing machine or examining the program in a debugger.
